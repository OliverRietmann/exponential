\section{Rational Exponents}
Now we are dealing with rational exponents $a^{\frac{m}{n}}$, where $a>0$ and $m,n\in\mathbb Z$ with $n\neq 0$.
These are closely related to roots.
\begin{example}
	If we naively apply the rule $(a^m)^n=a^{m\cdot n}$, we obtain
	\begin{equation*}
		a^{\frac{1}{2}}\cdot a^{\frac{1}{2}}
		=\left(a^{\frac{1}{2}}\right)^2
		=a^{\frac{1}{2}\cdot 2}=a^1=a.
	\end{equation*}
	This simply means
	\begin{equation*}
		a^{\frac{1}{2}}=\sqrt{a}.
	\end{equation*}
\end{example}
\begin{exercise}
	A third root $\sqrt[3]{a}$ of $a$ is a number that satisfies
	\begin{equation*}
		\sqrt[3]{a}\cdot\sqrt[3]{a}\cdot\sqrt[3]{a}=\left(\sqrt[3]{a}\right)^3=a.
	\end{equation*}
	Explain why it makes sense to define $a^{\frac{1}{3}}=\sqrt[3]{a}$.
\end{exercise}
\begin{solution*}
	This definition is compatible with the rule $(a^m)^n=a^{m\cdot n}$ since
	\begin{equation*}
		\left(a^{\frac{1}{3}}\right)^3=a^{\frac{1}{3}\cdot 3}=a^1=a.
	\end{equation*}
\end{solution*}
This remains valid for higher roots
\begin{equation*}
	\underbrace{a^{\frac{1}{n}}\cdot a^{\frac{1}{n}}\cdot\ldots\cdot a^{\frac{1}{n}}}_{n\text{--times}}
	=\left(a^{\frac{1}{n}}\right)^n=a^{\frac{1}{n}\cdot n}=a^1=a
\end{equation*}
and motivates the following definition of the $n$--th root
\begin{equation*}
	a^{\frac{1}{n}}=\sqrt[n]{a}.
\end{equation*}
We can thus extend the rules for exponents as follows.
\begin{tcolorbox}
	For bases $a,b\in\mathbb R$ and exponents $m,n\in\mathbb Z$, we have the following rules:
	\begin{multicols}{3}
		\centering
		\textbf{same base}\\
		\begin{align*}
			a^m\cdot a^n&=a^{m+n} \\[10pt]
			\frac{a^m}{a^n}&=a^{m-n} \\[10pt]
			\left(a^m\right)^n&=a^{m\cdot n} \\[10pt]
			\textcolor{blue}{a^{\frac{m}{n}}}&\textcolor{blue}{=\left(\sqrt[n]{a}\right)^m}
			\textcolor{blue}{=\sqrt[n]{a^m}}
		\end{align*}
		\vfill
		\columnbreak
		
		\textbf{same exponent}\\
		\begin{align*}
			\left(a\cdot b\right)^n&=a^n\cdot b^n \\[10pt]
			\left(\frac{a}{b}\right)^n&=\frac{a^n}{b^n} \\[10pt]
			\textcolor{blue}{\left(a\cdot b\right)^{\frac{1}{n}}}&\textcolor{blue}{=\sqrt[n]{a}\cdot\sqrt[n]{b}} \\[10pt]
			\textcolor{blue}{\left(\frac{a}{b}\right)^\frac{1}{n}}&
			\textcolor{blue}{=\frac{\sqrt[n]{a}}{\sqrt[n]{b}}}
		\end{align*}
		\vfill
		\columnbreak
		
		\textbf{definitions}\\
		\begin{align*}
			a^0&=1 \\[10pt]
			a^{-n}&=\frac{1}{a^n} \\[10pt]
			\textcolor{blue}{a^{\frac{1}{n}}}&\textcolor{blue}{=\sqrt[n]{a}}% \\[10pt]
			%\textcolor{blue}{a^{-\frac{1}{n}}}&\textcolor{blue}{=\frac{1}{\sqrt[n]{a}}}
		\end{align*}
		\vfill
	\end{multicols}
	Whenever negative powers are involved, the base must not be zero.
\end{tcolorbox}