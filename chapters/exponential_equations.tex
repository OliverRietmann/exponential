\section*{Exponential Equations}
We recall the basic laws for rational exponents:
\begin{tcolorbox}
	\begin{align*}
		a^x\cdot a^y&=a^{x+y}&\text{(same base)}\\
		a^x\cdot b^x&=\left(a\cdot b\right)^x&\text{(same exponent)}\\
		\left(a^x\right)^y&=a^{x\cdot y}&\text{(exponent of a power)}\\
		a^{-x}&=\frac{1}{a^x}&\text{(negative exponent)}\\
		a^{\frac{1}{n}}&=\sqrt[n]{a}&\text{(root)}
	\end{align*}
\end{tcolorbox}
These rules can be used to solve equations involving powers.
\begin{exercise}
	Determine all solutions:
	\begin{tasks}(3)
		\task $x^3=-64$
		\task $x^4=10^{-4}$
		\task $x^6=729$
		\task $x^3=1000$
		\task $x^{-6}=729$
		\task $x^{-\frac{5}{2}}=3^{\frac{5}{2}}$
	\end{tasks}
\end{exercise}
Now we consider equations where the unknown $x$ occurs in the exponent, so-called $\textit{exponential equations}$. We solve them with the following recipe:
\begin{tcolorbox}
	Solving exponential equations:
	\begin{enumerate}[Step 1:]
		\item Express both sides as powers with the same base.
		\item Compare the exponents.
	\end{enumerate}
\end{tcolorbox}
\begin{exercise}
	Solve for $x$:
	\begin{tasks}(4)
		\task $2^x=32$
		\task $5^x=25$
		\task $3^x=81$
		\task $7^x=1$
		\task $3^x=\frac{1}{3}$
		\task $2^x=\sqrt{2}$
		\task $5^x=\frac{1}{125}$
		\task $4^{x+1}=64$
	\end{tasks}
\end{exercise}
\begin{exercise}
	Solve for $x$:
	\begin{tasks}(4)
		\task $8^x=32$
		\task $4^x=\frac{1}{8}$
		\task $9^x=27$
		\task $25^x=\frac{1}{5}$
		\task $27^x=\frac{1}{9}$
		\task $16^x=\sqrt{32}$
		\task $4^{x+2}=128$
		\task $25^{1+x}=\frac{1}{125}$
	\end{tasks}
\end{exercise}
\begin{exercise}
	Solve for $x$, if possible:
	\begin{tasks}(3)
		\task $4^{2x+1}=8^{1-x}$
		\task $9^{2-x}=\left(\frac{1}{3}\right)^{2x+1}$
		\task $2^x\cdot 8^{1-x}=\frac{1}{4}$
		\task $3^{x+2}\cdot 9^x=27$
		\task $\left(\frac{1}{2}\right)^{x-1}\cdot 8^x=4^{-x}$
		\task $\left(\frac{1}{5}\right)^{x^2}\cdot 25^x=\frac{1}{125}$
	\end{tasks}
\end{exercise}
Now we consider more involved equations where the recipe cannot be applied directly.
\begin{example*}
	The previous approach fails for
	\begin{equation*}
		4^x+2^x-20=0
	\end{equation*}
	since we now have sums of powers. However, if we substitute $y=2^x$, we end up with a quadratic equation. We solve the latter and then reverse the substitution.
	\begin{align*}
		4^x+2^x-20&=0\\
		\left(2^x\right)^2+2^x-20&=0\\
		y^2+y-20&=0\quad\quad\leftarrow y=2^x\\
		y=4\text{ or }y&=-5\\
		2^x=4\text{ or }2^x&=-5\quad\leftarrow y=2^x
	\end{align*}
	The first equation yields $x=2$. The second one has no solution since $2^x>0$.
	We conclude that $x=2$ is the only solution.
\end{example*}

\begin{exercise}
	Determine all solutions:
	\begin{tasks}(3)
		\task $4^x-6\cdot 2^x+8=0$
		\task $4^x-2^x-2=0$
		\task $9^x-12\cdot 3^x+27=0$
		\task $9^x=3^x+6$
		\task $25^x-23\cdot 5^x-50=0$
		\task $49^x+1=2\cdot 7^x$
	\end{tasks}
\end{exercise}