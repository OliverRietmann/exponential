\section{Solving Equations}
We recall the basic laws for rational exponents:
\begin{tcolorbox}
	For bases $a,b\in\mathbb R$, exponents $x,y\in\mathbb Q$ and $n\in\mathbb N$, we have the following rules:
	\begin{multicols}{3}
		\centering
		\textbf{same base}\\
		\begin{align*}
			a^x\cdot a^y&=a^{x+y} \\[8pt]
			\frac{a^x}{a^y}&=a^{x-y} \\[8pt]
			\left(a^x\right)^y&=a^{x\cdot y}
		\end{align*}
		\vfill
		\columnbreak
		
		\textbf{same exponent}\\
		\begin{align*}
			\left(a\cdot b\right)^x&=a^x\cdot b^x \\[8pt]
			\left(\frac{a}{b}\right)^x&=\frac{a^x}{b^x}
		\end{align*}
		\vfill
		\columnbreak
		
		\textbf{definitions}\\
		\begin{align*}
			a^0&=1 \\[8pt]
			a^{-x}&=\frac{1}{a^x} \\[8pt]
			a^{\frac{1}{n}}&=\sqrt[n]{a}
		\end{align*}
		\vfill
	\end{multicols}
	Note that $x$ and $y$ can now be fractions of integers, for example $x=\frac{m}{n}$.
\end{tcolorbox}
\begin{exercise}
	\begin{tasks}
		\task Give an example where $\sqrt{a^2}\neq a$.
		\task Give an example where $\sqrt[n]{a}<0$.
		%\task Give an example where $x^n=a$ but $\sqrt[n]{a}\neq x$.
	\end{tasks}
\end{exercise}
\begin{exercise}
	Determine all solutions:
	\begin{tasks}(3)
		\task $x^3=-64$
		\task $x^4=10^{-4}$
		\task $x^6=729$
		\task $x^3=1000$
		\task $x^{-6}=729$
		\task $x^{-\frac{5}{2}}=3^{\frac{5}{2}}$
	\end{tasks}
\end{exercise}
%\annotation{6}
\begin{tcolorbox}
	\textbf{Exponential equations:} The unknown $x$ occurs in the exponent.
	\begin{itemize}
		\item exponential equations: $2^x=16$ and $3^{x+2}=\frac{1}{27}$
		\item \textbf{not} exponential equations: $x^2=16$ and $\left(x+2\right)^3=\frac{1}{27}$
	\end{itemize}
	Solving exponential equations:
	\begin{enumerate}[Step 1:]
		\item Express both sides as in terms of the same base.
		\item Compare the exponents.
	\end{enumerate}
\end{tcolorbox}
\pagebreak[4]
\begin{exampleenv}
\begin{example}
	\begin{minipage}{0.45\textwidth}
		\begin{align*}
			\textrm{a)}\quad 2^x&=16 \\
			2^x&=2^4 \\
			x&=4 \\
			\phantom{x+2}& \\
			\phantom{x+2}&
		\end{align*}
	\end{minipage}\hfill
	\begin{minipage}{0.45\textwidth}
		\begin{align*}
			\textrm{b)}\quad 3^{x+2}&=\frac{1}{27} \\
			3^{x+2}&=3^{-3} \\
			x+2&=-3 \\
			x&=-5
		\end{align*}
	\end{minipage}
\end{example}
%\annotation{10}
\end{exampleenv}
\begin{exercise}
	Solve for $x$.
	\begin{tasks}(4)
		\task $2^x=2$ \task $2^x=4$ \task $3^x=27$ \task $2^x=1$
		\task $2^x=\frac{1}{2}$ \task $3^x=\frac{1}{3}$ \task $2^x=\frac{1}{8}$ \task $2^{x+1}=8$
		\task $2^{x-2}=\frac{1}{4}$ \task $3^{x+1}=\frac{1}{27}$ \task $2^{x+1}=64$
		\task $2^{1-2x}=\frac{1}{2}$
	\end{tasks}
\end{exercise}
%\annotation{17}
%\begin{exercise}
%	Solve for $x$.
%	\begin{tasks}(4)
%		\task $2^x=32$
%		\task $5^x=25$
%		\task $3^x=81$
%		\task $7^x=1$
%		\task $3^x=\frac{1}{3}$
%		\task $2^x=\sqrt{2}$
%		\task $5^x=\frac{1}{125}$
%		\task $4^{x+1}=64$
%	\end{tasks}
%\end{exercise}
%\annotation{25}
\begin{exercise}
	Solve for $x$.
	\begin{tasks}(4)
		\task $8^x=32$
		\task $4^x=\frac{1}{8}$
		\task $9^x=27$
		\task $25^x=\frac{1}{5}$
		\task $27^x=\frac{1}{9}$
		\task $16^x=\sqrt{32}$
		\task $4^{x+2}=128$
		\task $25^{1+x}=\frac{1}{125}$
	\end{tasks}
\end{exercise}
%\annotation{25}
\begin{exercise}
	Solve for $x$, if possible:
	\begin{tasks}(3)
		\task $4^{2x+1}=8^{1-x}$
		\task $9^{2-x}=\left(\frac{1}{3}\right)^{2x+1}$
		\task $2^x\cdot 8^{1-x}=\frac{1}{4}$
		\task $3^{x+2}\cdot 9^x=27$
		\task $\left(\frac{1}{2}\right)^{x-1}\cdot 8^x=4^{-x}$
		\task $\left(\frac{1}{5}\right)^{x^2}\cdot 25^x=\frac{1}{125}$
	\end{tasks}
\end{exercise}
%\annotation{32}
Now we consider more involved equations where the recipe cannot be applied directly.
\begin{exampleenv}
\begin{example}
	The previous approach fails for
	\begin{equation*}
		4^x+2^x-20=0
	\end{equation*}
	since we now have sums of powers. However, if we substitute $y=2^x$, we end up with a quadratic equation. We solve the latter and then reverse the substitution.
	\begin{align*}
		4^x+2^x-20&=0\\
		\left(2^x\right)^2+2^x-20&=0\\
		y^2+y-20&=0\quad\quad\leftarrow y=2^x\\
		y=4\text{ or }y&=-5\\
		2^x=4\text{ or }2^x&=-5\quad\leftarrow y=2^x
	\end{align*}
	The first equation yields $x=2$. The second one has no solution since $2^x>0$.
	We conclude that $x=2$ is the only solution.
\end{example}
%\annotation{37}
\end{exampleenv}
\begin{exercise}
	Determine all solutions:
	\begin{tasks}(3)
		\task $4^x-6\cdot 2^x+8=0$
		\task $4^x-2^x-2=0$
		\task $9^x-12\cdot 3^x+27=0$
		\task $9^x=3^x+6$
		\task $25^x-23\cdot 5^x-50=0$
		\task $49^x+1=2\cdot 7^x$
	\end{tasks}
\end{exercise}
%\annotation{50}