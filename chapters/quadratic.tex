\begin{example}[quadratic growth] \label{ex:quadratic_growth}
	Consider a rectangle where the first edge has length $x>0$ and the second edge as length $a\cdot x$ for some $a>0$.
	The area of the rectangle is then given by $a\cdot x^2$.
	Thus, the function $f\left(x\right)=a\cdot x^2$ associates to the length $x$ of the first edge the area of the corresponding rectangle.
	We say, that the area grows \textit{quadratically} with the edge length.
	This is an example of \textit{quadratic growth}, see Figure~\ref{fig:linear_quadratic_growth}.
\end{example}
But what exactly characterizes linear and quadratic growth?
\begin{tcolorbox}
	Let $f\left(x\right)$ be a function.
	\begin{itemize}
		\item We say that $f$ grows \textit{linearly}, if $f\left(ax\right)=af\left(x\right)$ for all $a\in\mathbb R$.
		\item We say that $f$ grows \textit{quadratically}, if $f\left(ax\right)=a^2f\left(x\right)$ for all $a\in\mathbb R$.
	\end{itemize}
\end{tcolorbox}