\section*{Exponential Decay}
The \ce{^{14}C} isotope of carbon has a \textbf{half-live} of 5730 years, that is \textbf{every \ce{^{14}C} atom has a probability of 50\% to decay with the next 5730 years}.
Put differently, for any given an amount $a$ of \ce{^{14}C} atoms, we expect that after 5730 years, half of them will have decayed, so that we are left with roughly $\frac{a}{2}$ atoms.
After another 5730 years we will be left with half of $\frac{a}{2}$, i.e. with $\frac{a}{4}$.
We arrive at the following table.
\begin{figure}[ht]
	\centering
	\begin{tikzpicture}
		\matrix[matrix of math nodes,draw, column sep=1em,row sep=.5mm] (mx) {
			\textrm{year} & 0 & 5730 & 11460 & 17190 & 22920 & 28650 & 34380 \\
			\textrm{atoms} & \phantom{\frac{a}{1}}a\phantom{\frac{a}{1}} & \frac{a}{2} & \frac{a}{2^2} & \frac{a}{2^3} & \frac{a}{2^4} & \frac{a}{2^5} & \frac{a}{2^6} \\
		};
		\path[->,shorten >=2pt]
		\foreach \from/\to in {2/3,3/4,4/5,5/6,6/7,7/8} {
			([yshift=2mm]mx-1-\from.north) edge[bend left]
			node[above] {$\scriptstyle+5730$} ([yshift=2mm]mx-1-\to.north)
			([yshift=-2.5mm]mx-2-\from.south) edge[bend right]
			node[below] {$\scriptstyle\cdot\frac{1}{2}$} ([yshift=-2.5mm]mx-2-\to.south)
		};
		\foreach \x in {2,...,8}{
			\draw ([xshift=-1em]mx.north west -| mx-1-\x.west) -- ([xshift=-1em]mx.south west -| mx-1-\x.west);
		};
		\draw (mx.west) -- (mx.east);
	\end{tikzpicture}
\end{figure}
Suppose that at time zero, we have $a\in\mathbb N$ atoms of the \ce{^{14}C} isotope.
We want to find a function $f\left(x\right)$ that returns the expected number of \ce{^{14}C} atoms after $x$ years.
Put differently, the function $f\left(x\right)$ should yield exactly the values in the table above (where $x$ is the first row and $f\left(x\right)$ is the second row), this means
\begin{equation*}
	f\left(0\right)=a,\quad
	f\left(5730\right)=\frac{a}{2},\quad
	f\left(5730\cdot 2\right)=\frac{a}{2^2},\quad\ldots,\quad
	f\left(5730\cdot n\right)=\frac{a}{2^n}.
\end{equation*}
\begin{minipage}{0.48\textwidth}
	These conditions are satisfied by
	\begin{equation*}
		f\left(x\right)=a\left(\frac{1}{2}\right)^{\tfrac{x}{5730}},
	\end{equation*}
	because
	\begin{equation*}
		f\left(5730\cdot n\right)=a\left(\frac{1}{2}\right)^{\tfrac{5730\cdot n}{5730}}
		=a\left(\frac{1}{2}\right)^n
		=\frac{a}{2^n}.
	\end{equation*}
	Finally, this can be re-written as
	\begin{equation*}
		f\left(x\right)=a\cdot 2^{-x/5730}.
	\end{equation*}
	The graph of $f$ is shown on the left.
\end{minipage}\hfill
\begin{minipage}{0.48\textwidth}
		\centering
		\includegraphics[width=\textwidth]{images/decay}\hfill
		Decay of \ce{^{14}C} with $a=1024$ atoms.
\end{minipage}
	We could also ask, how many \ce{^{14}C} atoms are left after $x=\frac{5730}{2}=2865$ years
\begin{equation*}
	f\left(\tfrac{5730}{2}\right)=a\cdot 2^{-\frac{1}{2}}.
\end{equation*}
But what is the meaning of $2^{-\frac{1}{2}}$?
How can we multiply 2 by itself for $-\frac{1}{2}$ times?
\begin{exercise} \label{ex:c14}
	Suppose that at time zero we have $a=1024$ atoms of the \ce{^{14}C} isotope.
	How many years will it take until we expect only $64$ atoms to be left?
	\textit{Hint:} Note that $1024=2^{10}$ and $64=2^6$.
\end{exercise}
\begin{solution*}
	The number of \ce{^{14}C} atoms after $x$ years is given by
	\begin{equation*}
		f\left(x\right)=a\cdot 2^{-x/5730}=2^{10}\cdot 2^{-x/5730}=2^{-x/5730+10}.
	\end{equation*}
	We have to solve the equation $64=f\left(x\right)$, that is
	\begin{equation*}
		2^6=2^{-x/5730+10}\qquad\text{or}\qquad 6=-\frac{x}{5730}+10
	\end{equation*}
	upon comparing the exponents.
	Solving the second equation, we obtain
	\begin{equation*}
		x=4\cdot 5730=22920
	\end{equation*}
	years.
\end{solution*}
In fact, we have been using our rules of computation for the rational exponent $-\frac{x}{5730}$.
It turns out, that all the rules for exponents we know work more generally for real exponents.
\begin{tcolorbox}
	For exponents $m,n\in\mathbb R$, we have the following rules:
	\begin{multicols}{3}
		\centering
		\textbf{same base}\\
		\begin{align*}
			a^m\cdot a^n&=a^{m+n} \\[10pt]
			\frac{a^m}{a^n}&=a^{m-n} \\[10pt]
			\left(a^m\right)^n&=a^{m\cdot n}
		\end{align*}
		\vfill
		\columnbreak
		
		\textbf{same exponent}\\
		\begin{align*}
			\left(a\cdot b\right)^n&=a^n\cdot b^n \\[10pt]
			\left(\frac{a}{b}\right)^n&=\frac{a^n}{b^n}
		\end{align*}
		\vfill
		\columnbreak
		
		\textbf{definitions}\\
		\begin{align*}
			a^0&=1 \\[10pt]
			a^{-n}&=\frac{1}{a^n}
		\end{align*}
		\vfill
	\end{multicols}
	Whenever negative powers are involved, the base must not be zero.
\end{tcolorbox}
This means that we can re-write the function for radioactive decay of \ce{^{14}C}.
\begin{equation*}
	f\left(x\right)=a\cdot 2^{-x/5730}=2^{10}\cdot 2^{-x/5730}=2^{-x/5730+10}.
\end{equation*}

\begin{figure}[ht]
	\centering
	\includegraphics[width=0.48\textwidth]{images/exponentials_grow}\hfill
	\includegraphics[width=0.48\textwidth]{images/exponentials_decay}
	\caption{Graph of exponentially growing functions with $a=1$.}
	\label{fig:graphs}
\end{figure}
\begin{exercise} \label{ex:graphs}
	Take a look at Figure~\ref{ex:graphs} and come up with some properties about functions of the form $f\left(x\right)=a\cdot b^x$.
\end{exercise}
\begin{solution*}
	We have $f\left(0\right)=a$, so $a$ is always the value of $f$ at zero.
	The larger $b$ the faster $f\left(x\right)$ will grow.
	If $0<b<1$, then $f\left(x\right)$ is even decreasing.
	Moreover, replacing $b$ by $\frac{1}{b}$ seems to mirror the graph about the $y$--axis.
\end{solution*}
\textcolor{red}{TODO: Radioactive decay, half life, motivation of rational exponents}