\section{Exponential Function}
\textcolor{red}{TODO: Euler's number e, properties, exercises}\\
The \ce{^{14}C} isotope of carbon has a \textbf{half-live} of 5730 years, that is \textbf{every \ce{^{14}C} atom has a probability of 50\% to decay with the next 5730 years}.
Put differently, for any given an amount $a$ of \ce{^{14}C} atoms, we expect that after 5730 years, half of them will have decayed, so that we are left with roughly $\frac{a}{2}$ atoms.
After another 5730 years we will be left with half of $\frac{a}{2}$, i.e. with $\frac{a}{4}$.
We arrive at the following table.
\begin{figure}[ht]
	\centering
	\begin{tikzpicture}
		\matrix[matrix of math nodes,draw, column sep=1em,row sep=.5mm] (mx) {
			\textrm{year} & 0 & 5730 & 11460 & 17190 & 22920 & 28650 & 34380 \\
			\textrm{atoms} & \phantom{\frac{a}{1}}a\phantom{\frac{a}{1}} & \frac{a}{2} & \frac{a}{2^2} & \frac{a}{2^3} & \frac{a}{2^4} & \frac{a}{2^5} & \frac{a}{2^6} \\
		};
		\path[->,shorten >=2pt]
		\foreach \from/\to in {2/3,3/4,4/5,5/6,6/7,7/8} {
			([yshift=2mm]mx-1-\from.north) edge[bend left]
			node[above] {$\scriptstyle+5730$} ([yshift=2mm]mx-1-\to.north)
			([yshift=-2.5mm]mx-2-\from.south) edge[bend right]
			node[below] {$\scriptstyle\cdot\frac{1}{2}$} ([yshift=-2.5mm]mx-2-\to.south)
		};
		\foreach \x in {2,...,8}{
			\draw ([xshift=-1em]mx.north west -| mx-1-\x.west) -- ([xshift=-1em]mx.south west -| mx-1-\x.west);
		};
		\draw (mx.west) -- (mx.east);
	\end{tikzpicture}
\end{figure}
the number of remaining atoms after $x$ is given by the function
\begin{equation*}
	f\left(x\right)=a\cdot 2^{-\frac{x}{5730}}
\end{equation*}
\begin{minipage}{0.48\textwidth}
	\centering
	%\includegraphics[width=\textwidth]{images/decay}
\end{minipage}\hfill
\begin{minipage}{0.48\textwidth}
	\centering
	%\includegraphics[width=\textwidth]{images/kriener_exponential}
\end{minipage}
\begin{tcolorbox}
	An \textbf{exponential function} is a function that can be written as
	\begin{equation*}
		f\left(x\right)=a\cdot b^x
	\end{equation*}
	for some $a\in\mathbb R\setminus\left\{0\right\}$ and some $b>0$.
	We call \ldots
	\begin{itemize}
		\item[] \ldots$a$ the \textbf{initial value} and \ldots
		\item[] \ldots$b$ the \textbf{growth factor} (even if $b<1$).
	\end{itemize}
\end{tcolorbox}
\begin{exercise}
	Show that
	\begin{equation*}
		f\left(x\right)=a\cdot 2^{-\frac{x}{5730}}
	\end{equation*}
	is an exponential function, i.e. find $b$.
	Why do you think $a$ is called initial value?
\end{exercise}
\pagebreak[1]
For a linear function $g\left(x\right)=a\cdot x+b$, we have
\begin{equation*}
	g\left(x+n\right)=a\cdot\left(x+n\right)+b=a\cdot x+b+a\cdot n=g\left(x\right)+a\cdot n.
\end{equation*}
Hence advancing on the $x$ axis by $n$ changes the function value by \textbf{adding} $a\cdot n$.
On the other hand, for an exponential function $f(x)=ab^x$, we have
\begin{equation*}
	f\left(x+n\right)=a\cdot b^{x+n}=a\cdot b^x\cdot b^n=f\left(x\right)\cdot b^n.
\end{equation*}
Advancing on the $x$ axis by $n$ changes the function value by the \textbf{factor} $b^n$.
\begin{exercise}
	These tables contain the values of a function $f\left(x\right)=a\cdot b^x$.
	Complete the tables and determine $a$ and $b$.
	\begin{center}
	\begin{tikzpicture}
		\node at (-5,0) {a)};
		\matrix[matrix of math nodes,draw, column sep=1em,row sep=.5mm] (mx) {
			x & 0 & 1 & 2 & 3 & 4 & 5 \\
			f\left(x\right) & \frac{243}{32} & \frac{8}{27} & \frac{2}{3} & \frac{3}{2} & \frac{27}{8} & \frac{243}{32} \\
		};
		\path[->,shorten >=2pt]
		\foreach \from/\to in {2/3,3/4,4/5,5/6,6/7} {
			([yshift=2mm]mx-1-\from.north) edge[bend left]
			node[above] {$\scriptstyle+1$} ([yshift=2mm]mx-1-\to.north)
			([yshift=-2.5mm]mx-2-\from.south) edge[bend right]
			node[below] {$\scriptstyle\times\frac{9}{4}$} ([yshift=-2.5mm]mx-2-\to.south)
		};
		\foreach \x in {2,...,7}{
			\draw ([xshift=-1em]mx.north west -| mx-1-\x.west) -- ([xshift=-1em]mx.south west -| mx-1-\x.west);
		};
		\draw (mx.west) -- (mx.east);
	\end{tikzpicture}
	\begin{tikzpicture}
		\node at (-5,0) {b)};
		\matrix[matrix of math nodes,draw, column sep=1em,row sep=.5mm] (mx) {
			x & -8 & 1 & 4 & 5 & 6 & 7 \\
			f\left(x\right) &  &  & \frac{3}{2} & & 6 & \\
		};
%		\path[->,shorten >=2pt]
%		\foreach \from/\to in {2/3,3/4,4/5,5/6,6/7} {
%			([yshift=2mm]mx-1-\from.north) edge[bend left]
%			node[above] {$\scriptstyle+1$} ([yshift=2mm]mx-1-\to.north)
%			([yshift=-2.5mm]mx-2-\from.south) edge[bend right]
%			node[below] {$\scriptstyle\times\frac{9}{4}$} ([yshift=-2.5mm]mx-2-\to.south)
%		};
		\foreach \x in {2,...,7}{
			\draw ([xshift=-1em]mx.north west -| mx-1-\x.west) -- ([xshift=-1em]mx.south west -| mx-1-\x.west);
		};
		\draw (mx.west) -- (mx.east);
	\end{tikzpicture}
	\end{center}
	\renewcommand{\arraystretch}{2.5}
	\setlength{\tabcolsep}{15pt}
	\begin{tabular}{|c|c|c|c|c|c|c|}\hline
		$x$ & 0 & 1 & 2 & 3 & 4 & 5 \\ \hline
		$f\left(x\right)$ & & & $\dfrac{2}{3}$ & $\dfrac{3}{2}$ & & \\ \hline
	\end{tabular}
\end{exercise}
\begin{exercise}
	Consider two points $P=\left(2,12\right)$ and $Q=\left(10,3072\right)$.
	\begin{tasks}
		\task Find the linear function $g\left(x\right)=a\cdot x+b$ that passes through these points.
		\task Find the exponential function $f\left(x\right)=a\cdot b^x$ that passes through these points.
		\task Given two points, is there always a linear function passing through these points?
		\task Given two points, is there always an exponential function passing through these points?
	\end{tasks}
\end{exercise}
%\begin{exercise}
%	Consider again the function
%	\begin{equation*}
%		f\left(x\right)=a\cdot 2^{-\frac{x}{5730}},
%	\end{equation*}
%	where $a$ is now unknown.
%	Suppuse that at time $x=$
%\end{exercise}