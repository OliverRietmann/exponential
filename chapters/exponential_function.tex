\section{Exponential Function}
The population of a village is growing by 6.25\% every year.
At year zero, the population consists of 4096 people.
This yields the following table (rounded to integers) for the expected population after $x$ years.
\begin{figure}[ht]
	\centering
	\begin{tikzpicture}
		\matrix[matrix of math nodes,draw, column sep=1em,row sep=.5mm] (mx) {
			\textrm{year x} & 0 & 1 & 2 & 3 & 4 & 5 & 6 \\
			\textrm{population $f(x)$} & 4096 & 4352 & 4624 & 4913 & 5220 & 5546 & 5893 \\
		};
		\path[->,shorten >=2pt]
		\foreach \from/\to in {2/3,3/4,4/5,5/6,6/7,7/8} {
			([yshift=2mm]mx-1-\from.north) edge[bend left]
			node[above] {$\scriptstyle+1$} ([yshift=2mm]mx-1-\to.north)
			([yshift=-2.5mm]mx-2-\from.south) edge[bend right]
			node[below] {$\scriptstyle\cdot\frac{17}{16}$} ([yshift=-2.5mm]mx-2-\to.south)
		};
		\foreach \x in {2,...,8}{
			\draw ([xshift=-1em]mx.north west -| mx-1-\x.west) -- ([xshift=-1em]mx.south west -| mx-1-\x.west);
		};
		\draw (mx.west) -- (mx.east);
	\end{tikzpicture}
\end{figure}
The population size as a function of the number of years $x$ is given by
\begin{equation*}
	f\left(x\right)=4096\cdot\left(\frac{17}{16}\right)^x.
\end{equation*}
%\begin{minipage}{0.48\textwidth}
%	\centering
%	%\includegraphics[width=\textwidth]{images/decay}
%\end{minipage}\hfill
%\begin{minipage}{0.48\textwidth}
%	\centering
%	%\includegraphics[width=\textwidth]{images/kriener_exponential}
%\end{minipage}
\begin{tcolorbox}
	An \textbf{exponential function} is a function that can be written as
	\begin{equation*}
		f\left(x\right)=a\cdot b^x
	\end{equation*}
	for some $a\in\mathbb R\setminus\left\{0\right\}$ and some $b>0$.
	We call \ldots
	\begin{itemize}
		\item[] \ldots$a$ the \textbf{initial value} and \ldots
		\item[] \ldots$b$ the \textbf{growth factor} (even if $b<1$).
	\end{itemize}
\end{tcolorbox}
%\begin{exercise}
%	Show that
%	\begin{equation*}
%		f\left(x\right)=a\cdot 2^{-\frac{x}{5730}}
%	\end{equation*}
%	is an exponential function, i.e. find $b$.
%	Why do you think $a$ is called initial value?
%\end{exercise}
\begin{itemize}
	\item For a linear function $g\left(x\right)=a\cdot x+b$, we have
	\begin{equation*}
		g\left(x+n\right)=a\cdot\left(x+n\right)+b=a\cdot x+b+a\cdot n=g\left(x\right)+a\cdot n.
	\end{equation*}
	Hence advancing on the $x$ axis by $n$ changes the function value by \textbf{adding} $a\cdot n$.
	\item On the other hand, for an exponential function $f(x)=a\cdot b^x$, we have
	\begin{equation*}
		f\left(x+n\right)=a\cdot b^{x+n}=a\cdot b^x\cdot b^n=f\left(x\right)\cdot b^n.
	\end{equation*}
	Advancing on the $x$ axis by $n$ changes the function value by the \textbf{factor} $b^n$.
\end{itemize}
\begin{exercise}
	These tables contain the values of a function $f\left(x\right)=a\cdot b^x$.
	Complete the tables, determine $a$ and $b$, and draw the graph of $f\left(x\right)$.
	\renewcommand{\arraystretch}{2}
	\setlength{\tabcolsep}{15pt}
	%\newcolumntype{C}[1]{>{\centering\let\newline\\\arraybackslash\hspace{0pt}}m{#1}}
	\begin{tasks}(1)
		\task \hspace{10pt} \begin{tabular}{|c|c|c|c|c|c|c|}\hline
			$x$ & 0 & 1 & 2 & 3 & 4 & 5 \\ \hline
			$f\left(x\right)$ & & $\dfrac{2}{3}$ & $\dfrac{3}{2}$ & $\dfrac{27}{8}$ & &  \\[6pt] \hline
		\end{tabular} \vspace{30pt}
		\task \hspace{10pt} \begin{tabular}{|c|c|c|c|c|c|c|c|c|c|}\hline
			$x$ & $\smash{-}8$ & 1 & 4 & 5 & 6 & 7 & 8 & 9 & 10 \\ \hline
			$f\left(x\right)$ &  &  & $\dfrac{3}{2}$ & & 6 & & & & \\[6pt] \hline
		\end{tabular} \vspace{15pt}
		\task \hspace{10pt} \begin{tabular}{|c|c|c|c|c|c|c|}\hline
			$x$ & 0 & 3 & 4 & 5 & 6 & 10 \\ \hline
			$f\left(x\right)$ &  & $\dfrac{9}{8}$ &  & & $\dfrac{1}{3}$ & \\[6pt] \hline
		\end{tabular}
	\end{tasks}
\end{exercise}
%\begin{exercise}
%	Which function belongs to which plot?\\[10pt]
%	\begin{minipage}{0.48\textwidth}
%		\centering
%		\includegraphics[width=\textwidth]{images/choose_exp_1}
%	\end{minipage}
%	\begin{minipage}{0.48\textwidth}
%		\centering
%		\includegraphics[width=\textwidth]{images/choose_exp_2}
%	\end{minipage}
%\end{exercise}
\begin{exercise}
	The $\left(2,12\right)$ and $\left(10,3072\right)$ lie on the function $f\left(x\right)$.
	Determine $a$ and $b$ if
	\begin{tasks}(2)
		\task $f\left(x\right)=ax+b$
		\task $f\left(x\right)=a\cdot b^x$
	\end{tasks}
\end{exercise}
\pagebreak
\begin{exercise}
	Suppose a forest is growing exponentially.
	Today, the area of the forest is $72\,342\ m^2$.
	Twelve years ago, it was $48\,128\ m^2$.
	\begin{tasks}(2)
		\task How large was the forest 5 years ago?
		\task How large will it be in 7 years?
	\end{tasks}
\end{exercise}
\begin{exercise}
	A population of bacteria grows exponentially.
	At 08:00 there were 2300 bacteria and at 12:00, there were 36200 bacteria.
	How many were there at
	\begin{tasks}(4)
		\task 09:00
		\task 10:00
		\task 11:00
		\task 13:30
	\end{tasks}
\end{exercise}
\begin{exercise}
	Let $f\left(x\right)$ describe the number of citizens of a town at time $t$.
	Find a formula for $f\left(x\right)$ if:
	\begin{tasks}
		\task $f\left(0\right)=5000$ and the population doubles every 20 years.
		\task $f\left(10\right)=1000$ and the population drops to half the size every 50 years.
		\task What is the growth factor for $\Delta x=20$ years in a)?
		\task What is the growth factor for $\Delta x=50$ years in b)?
	\end{tasks}
\end{exercise}
\begin{definition*}
	Let $f\left(x\right)$ be an exponential function.
	\begin{itemize}
		\item If $f\left(x\right)$ is increasing: The time span $\Delta x$ of the growth factor 2 is called \textbf{doubling time}.
		\item If $f\left(x\right)$ is decreasing: The time span $\Delta x$ of the growth factor $\frac{1}{2}$ is called \textbf{half-life}.
	\end{itemize}
\end{definition*}
%\begin{definition*}
%	The last exercise was about \textbf{doubling time} and \textbf{half-life}:
%	\begin{itemize}
%	\item The \textbf{doubling time} of an exponentially growing function $f\left(x\right)$ is the time span $\Delta x$ after which the function value doubles i.e.
%	\begin{equation*}
%		f\left(x+\Delta x\right)=f\left(x\right)\cdot 2.
%	\end{equation*}
%	\item The \textbf{half-life} of an exponentially decaying function $f\left(x\right)$ is the time span $\Delta x$ after which the function value drops to half, i.e.
%	\begin{equation*}
%		f\left(x+\Delta x\right)=f\left(x\right)\cdot\frac{1}{2}.
%	\end{equation*}
%	\end{itemize}
%\end{definition*}
\begin{example}
	In the previous exercise, the doubling time in a) is $\Delta x=20$ years and the half-life in b) is $\Delta x=50$ years.
\end{example}
\begin{exercise}
	Determine half-life or doubling time of the following exponential functions.
	\begin{tasks} (4)
		\task $f\left(x\right)=3\cdot\left(\dfrac{1}{2}\right)^x$
		\task $f\left(x\right)=2\cdot 2^x$
		\task $f\left(x\right)=5\cdot 4^x$
		\task $f\left(x\right)=4\cdot\left(\dfrac{1}{8}\right)^x$
	\end{tasks}
\end{exercise}
\begin{exercise}
	Consider an exponential function such that the step $\Delta x=5$ has an associated
	\begin{tasks}
		\task growth factor of $\dfrac{1}{8}$. What is the half-life?
		\task growth factor of $64$. What is the doubling time?
	\end{tasks}
\end{exercise}
\begin{exercise}
	The following two points lie on the function $f\left(x\right)=a\cdot b^x$.
	Determine $a$ and $b$.
	\begin{tasks}(2)
		\task $\left(-2,20\right)$ and $\left(3,\frac{3}{5}\right)$
		\task $\left(0,2\right)$ and $\left(2,18\right)$
		\task $\left(-4.1,0.004\right)$ and $\left(1.9,0.5\right)$
	\end{tasks}
\end{exercise}
%\begin{exercise}
%	Consider two points $P=\left(2,12\right)$ and $Q=\left(10,3072\right)$.
%	\begin{tasks}
%		\task Find the linear function $g\left(x\right)=a\cdot x+b$ that passes through these points.
%		\task Find the exponential function $f\left(x\right)=a\cdot b^x$ that passes through these points.
%		\task Given two points, is there always a linear function passing through these points?
%		\task Given two points, is there always an exponential function passing through these points?
%	\end{tasks}
%\end{exercise}
%\begin{exercise}
%	Consider again the function
%	\begin{equation*}
%		f\left(x\right)=a\cdot 2^{-\frac{x}{5730}},
%	\end{equation*}
%	where $a$ is now unknown.
%	Suppuse that at time $x=$
%\end{exercise}
%The \ce{^{14}C} isotope of carbon has a \textbf{half-live} of 5730 years, that is \textbf{every \ce{^{14}C} atom has a probability of 50\% to decay with the next 5730 years}.
%Put differently, for any given an amount $a$ of \ce{^{14}C} atoms, we expect that after 5730 years, half of them will have decayed, so that we are left with roughly $\frac{a}{2}$ atoms.
%After another 5730 years we will be left with half of $\frac{a}{2}$, i.e. with $\frac{a}{4}$.
%We arrive at the following table.
%\begin{figure}[ht]
%	\centering
%	\begin{tikzpicture}
%		\matrix[matrix of math nodes,draw, column sep=1em,row sep=.5mm] (mx) {
%			\textrm{year} & 0 & 5730 & 11460 & 17190 & 22920 & 28650 & 34380 \\
%			\textrm{atoms} & \phantom{\frac{a}{1}}a\phantom{\frac{a}{1}} & \frac{a}{2} & \frac{a}{2^2} & \frac{a}{2^3} & \frac{a}{2^4} & \frac{a}{2^5} & \frac{a}{2^6} \\
%		};
%		\path[->,shorten >=2pt]
%		\foreach \from/\to in {2/3,3/4,4/5,5/6,6/7,7/8} {
%			([yshift=2mm]mx-1-\from.north) edge[bend left]
%			node[above] {$\scriptstyle+5730$} ([yshift=2mm]mx-1-\to.north)
%			([yshift=-2.5mm]mx-2-\from.south) edge[bend right]
%			node[below] {$\scriptstyle\cdot\frac{1}{2}$} ([yshift=-2.5mm]mx-2-\to.south)
%		};
%		\foreach \x in {2,...,8}{
%			\draw ([xshift=-1em]mx.north west -| mx-1-\x.west) -- ([xshift=-1em]mx.south west -| mx-1-\x.west);
%		};
%		\draw (mx.west) -- (mx.east);
%	\end{tikzpicture}
%\end{figure}
%the number of remaining atoms after $x$ is given by the function
%\begin{equation*}
%	f\left(x\right)=a\cdot 2^{-\frac{x}{5730}}
%\end{equation*}