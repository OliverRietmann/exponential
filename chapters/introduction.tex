\section*{Introduction: Exponential Growth}
%Before we look at exponential growth, let us recall the familiar concept of \textit{linear growth}.
\begin{example}[linear growth] \label{ex:linear_growth}
	Suppose we have a bank account with 50 CHF at the beginning of year zero.
	At the end of every year, we deposit 20 CHF.
	How much money are we going to have at the beginning of year $x$?
	The answer clearly is $20x+50$ CHF.
	This is visulaized in Figure~\ref{ex:linear_growth} (left).
	Also, we can fill the amount of money at the beginning of each year into a table.
%	\begin{table}[ht]
%		\centering
%		\begin{tabular}{|l|c|c|c|c|c|c|c|c|c|} \hline
%			year & 0 & 1 & 2 & 3 & 4 & 5 & 6 & 7 & 8 \\ \hline
%			CHF & 50 & 70 & 90 & 110 & 130 & 150 & 170 & 190 & 210 \\ \hline
%		\end{tabular}
%		\label{tab:linear_growth}
%	\end{table}
	\begin{figure}[ht]
	\centering
	\begin{tikzpicture}
		\matrix[matrix of math nodes,draw, column sep=1em,row sep=.5mm] (mx) {
			\textrm{year} & 0 & 1 & 2 & 3 & 4 & 5 & 6 & 7 & 8 \\
			\textrm{CHF} & 50 & 70 & 90 & 110 & 130 & 150 & 170 & 190 & 210 \\
		};
		\path[->,shorten >=2pt]
		\foreach \from/\to in {2/3,3/4,4/5,5/6,6/7,7/8,8/9,9/10} {
			([yshift=2mm]mx-1-\from.north) edge[bend left]
			node[above] {$\scriptstyle+1$} ([yshift=2mm]mx-1-\to.north)
			([yshift=-2.5mm]mx-2-\from.south) edge[bend right]
			node[below] {$\scriptstyle+20$} ([yshift=-2.5mm]mx-2-\to.south)
		};
		\foreach \x in {2,...,10}{
			\draw ([xshift=-1em]mx.north west -| mx-1-\x.west) -- ([xshift=-1em]mx.south west -| mx-1-\x.west);
		};
		\draw (mx.west) -- (mx.east);
	\end{tikzpicture}
	\end{figure}
\end{example}
\begin{exercise} \label{ex:exponential_growth}
	Suppose we have again a bank account with 50 CHF at the beginning of year zero.
	But now, we will not deposit any additional money.
	Instead, the bank pays us an intrest of 20\% at the end of each year.
	So if we have 100 CHF on our bank account during a certain year, then the bank will pay us 20 CHF (i.e. 20\%) at the end of that year.
	\begin{tasks}
		\task How much money are we going to have at the beginning of year $x$? Create and fill a table analogusly to the one in Example~\ref{ex:linear_growth}.
		\task Find a function $f\left(x\right)$ that returns the amount of money at the end of year $x$.
		\task Draw the graph of the function $f\left(x\right)$ and compare it to the one in Example~\ref{ex:linear_growth}.
	\end{tasks}
\end{exercise}
\begin{solution*}
	At the end of year zero, we have 50 CHF on our bank account.
	Consequently, at the beginning of year one, we will have $60=50\cdot 1.2$ CHF.
	At the end of year two, we have $72=60\cdot 1.2$ CHF.
	This can also bewritten as $72=50\cdot 1.2^2$ CHF.
	Following this scheme, we arrive at the solution given below.
	\begin{enumerate}[a)]
		\item The table is given by
%		\begin{tabular}{|l|c|c|c|c|c|c|c|c|c|} \hline
%			year & 0 & 1 & 2 & 3 & 4 & 5 & 6 & 7 & 8 \\ \hline
%			CHF & 50 & 60 & 72 & 86.4 & 103.68 & 124.416 & 149.2992 & 179.15904 & 214.990848 \\ \hline
%		\end{tabular}\\
		\begin{figure}[ht]
			\centering
			\begin{tikzpicture}
				\matrix[matrix of math nodes,draw, column sep=1em,row sep=.5mm] (mx) {
					\textrm{year} & 0 & 1 & 2 & 3 & 4 & 5 & 6 & 7 & 8 \\
					\textrm{CHF} & 50 & 60 & 72 & 86.4 & 103.68 & 124.416 & 149.2992 & 179.15904 & 214.990848 \\
				};
				\path[->,shorten >=1pt]
				\foreach \from/\to in {2/3,3/4,4/5,5/6,6/7,7/8,8/9,9/10} {
					([yshift=2mm]mx-1-\from.north) edge[bend left]
					node[above] {$\scriptstyle+1$} ([yshift=2mm]mx-1-\to.north)
					([yshift=-2.5mm]mx-2-\from.south) edge[bend right]
					node[below] {$\scriptstyle\cdot 1.2$} ([yshift=-2.5mm]mx-2-\to.south)
				};
				\foreach \x in {2,...,10}{
					\draw ([xshift=-\x-15]mx.north west -| mx-1-\x.west) -- ([xshift=-\x-15]mx.south west -| mx-1-\x.west);
				};
				\draw (mx.west) -- (mx.east);
			\end{tikzpicture}
		\end{figure}
		\item The function is given by $f\left(x\right)=50\cdot 1.2^x$ CHF.
		\item The graph of $f\left(x\right)$ is given in Figure~\ref{fig:linear_exponential_growth} (right).
	\end{enumerate}
\end{solution*}
\begin{figure}[ht]
	\centering
	\includegraphics[width=0.45\textwidth]{images/linear_growth}\hfill
	\includegraphics[width=0.45\textwidth]{images/quadratic_growth}
	\caption{An example of linear growth (left) and exponential growth (right).}
	\label{fig:linear_exponential_growth}
\end{figure}
\begin{tcolorbox}
	We say that a function $f\left(x\right)$ grows \ldots
	\begin{itemize}
		\item[] \ldots\ \textbf{linearly}, if $f\left(x\right)=ax+b$ for some $a>0$, and \ldots
		\item[] \ldots\ \textbf{exponentially}, if $f\left(x\right)=a\cdot b^x$ for some $a>0$ and $b>1$.
	\end{itemize}
\end{tcolorbox}
\begin{example}
	The function in Exercise~\ref{ex:exponential_growth} grows \textit{exponentially} with $a=50$ CHF and $b=1.2$.
\end{example}
\begin{exercise} \label{ex:first_exponential_equation}
	This time, we have only 20 CHF on our bank account at year zero.
	The bank still pays in intrest of 20\% at the end of each year.
	How many years will it take until we have more than 500 CHF on our bank account?
\end{exercise}
\begin{solution*}
	At the beginning of year $x$, we have $f\left(x\right)=80\cdot 1.2^x$ CHF on our bank account.
	We have $f\left(10\right)=495.34$ CHF and $f\left(11\right)=594.41$ CHF (both rounded to two digits).
	Hence it takes 11 years.
\end{solution*}

In Exercise~\ref{ex:first_exponential_equation}, we have solved our first \textit{exponential equation}, i.e. an equation where the unknown $x$ occurs in the exponent.
In the next chapter, we develop a more sytematic approach to solve exponential equations.