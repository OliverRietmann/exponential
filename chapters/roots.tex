\section{Roots and Rational Exponents}
Now we are dealing with rational exponents $a^{\frac{m}{n}}$, where $a>0$ and $m,n\in\mathbb Z$ with $n\neq 0$.
These are closely related to roots.
\begin{tcolorbox}
	Let $a\in\mathbb R$ and $n\in\mathbb N$. The expression
	\begin{equation*}
		\sqrt[n]{a}=b\qquad\text{means}\qquad a=b^n
	\end{equation*}
	and we say $b$ is an \textbf{\textit{n}th root} of $a$.
	\textbf{If \textit{n} is even}\ldots
	\begin{itemize}
		\item \ldots an \textit{n}th root exists if and only if $a\geq 0$.
		\item \ldots $-b$ is always another \textit{n}th root.
		\item \ldots we only denote the nonnegative root by $\sqrt[n]{a}$ and call it the \textbf{principal} root.
	\end{itemize}
\end{tcolorbox}
\begin{example}
	We have
	\begin{equation*}
		2^2=4\qquad\text{and}\qquad\left(-2\right)^2=4.
	\end{equation*}
	This means that both $2$ and $-2$ are square roots of $4$.
	However, only $2$ is called the \textbf{principal} square root of $4$.
	Hence we write $\sqrt{4}=2$ but we do \textbf{not} write $\sqrt{4}=-2$.
\end{example}
\begin{tcolorbox}
	Let $n\in\mathbb N$ and $a\in\mathbb R$. Then
	\begin{align*}
		\sqrt[n]{a^n}=a&\qquad\text{if }n\text{ is odd} \\
		\sqrt[n]{a^n}=\lvert a\rvert&\qquad\text{if }n\text{ is even}
	\end{align*}
\end{tcolorbox}
\begin{example}
	The expression $\sqrt{\left(-2\right)^2}$ is by definition the positive number $b$ such that 
	\begin{equation*}
		\left(-2\right)^2=b^2.
	\end{equation*}
	The only positive solution is $b=2$ and thus
	\begin{equation*}
		\sqrt{\left(-2\right)^2}=\lvert -2\rvert=2.
	\end{equation*}
\end{example}
\begin{tcolorbox}
	Let $m,n\in\mathbb N$ and $a,b\in\mathbb R$. Then
	\begin{equation*}
		\sqrt[n]{a\cdot b}=\sqrt[n]{a}\cdot\sqrt[n]{b}\qquad\text{and}\qquad
		\sqrt[n]{\frac{a}{b}}=\frac{\sqrt[n]{a}}{\sqrt[n]{b}}\qquad\text{and}\qquad
		\sqrt[n]{a^m}=\left(\sqrt[n]{a}\right)^m,
	\end{equation*}
	with $b\neq 0$ in the equation in the middle.
\end{tcolorbox}
\begin{example}
	If we naively apply the rule $(a^m)^n=a^{m\cdot n}$, we obtain
	\begin{equation*}
		a^{\frac{1}{2}}\cdot a^{\frac{1}{2}}
		=\left(a^{\frac{1}{2}}\right)^2
		=a^{\frac{1}{2}\cdot 2}=a^1=a.
	\end{equation*}
	This simply means
	\begin{equation*}
		a^{\frac{1}{2}}=\sqrt{a},
	\end{equation*}
	where we have assumed $a\geq0$.
	The reason is that the square of a number always yields a positive number.
	For example, $b^2=-1$ has no solution since $b^2>0$ even for negative $b$.
\end{example}
\begin{exercise}
	A third root $\sqrt[3]{a}$ of $a$ is a number that satisfies
	\begin{equation*}
		\sqrt[3]{a}\cdot\sqrt[3]{a}\cdot\sqrt[3]{a}=\left(\sqrt[3]{a}\right)^3=a.
	\end{equation*}
	Explain why it makes sense to define $a^{\frac{1}{3}}=\sqrt[3]{a}$.
\end{exercise}
\begin{solution*}
	This definition is compatible with the rule $(a^m)^n=a^{m\cdot n}$ since
	\begin{equation*}
		\left(a^{\frac{1}{3}}\right)^3=a^{\frac{1}{3}\cdot 3}=a^1=a.
	\end{equation*}
\end{solution*}
This remains valid for higher roots
\begin{equation*}
	\underbrace{a^{\frac{1}{n}}\cdot a^{\frac{1}{n}}\cdot\ldots\cdot a^{\frac{1}{n}}}_{n\text{--times}}
	=\left(a^{\frac{1}{n}}\right)^n=a^{\frac{1}{n}\cdot n}=a^1=a
\end{equation*}
and motivates the following definition of the $n$--th root
\begin{equation*}
	a^{\frac{1}{n}}=\sqrt[n]{a}.
\end{equation*}
We can thus extend the rules for exponents as follows.
\begin{tcolorbox}
	For bases $a,b>0$ and exponents $m,n\in\mathbb Z\setminus\left\{0\right\}$, we have the following rules:
	\begin{multicols}{3}
		\centering
		\textbf{same base}\\
		\begin{align*}
			%a^{\frac{1}{m}}\cdot a^\frac{1}{n}&=a^{\frac{1}{m}+\frac{1}{n}} \\[10pt]
			a^{\frac{m}{n}}&=\left(\sqrt[n]{a}\right)^m
			=\sqrt[n]{a^m}
		\end{align*}
		\vfill
		\columnbreak
		
		\textbf{same exponent}\\
		\begin{align*}
			\left(a\cdot b\right)^{\frac{1}{n}}&=\sqrt[n]{a}\cdot\sqrt[n]{b}% \\[10pt]
			%\left(\frac{a}{b}\right)^\frac{1}{n}&=\frac{\sqrt[n]{a}}{\sqrt[n]{b}}
		\end{align*}
		\vfill
		\columnbreak
		
		\textbf{definitions}\\
		\begin{align*}
			a^{\frac{1}{n}}&=\sqrt[n]{a}% \\[10pt]
			%\textcolor{blue}{a^{-\frac{1}{n}}}&\textcolor{blue}{=\frac{1}{\sqrt[n]{a}}}
		\end{align*}
		\vfill
	\end{multicols}
\end{tcolorbox}
\begin{example}
	Write the term on the left as a power of two.
	\begin{tasks}(3)
		\task $\sqrt[3]{2}=2^{\frac{1}{3}}$
		\task $\dfrac{1}{\sqrt{2}}=\dfrac{1}{2^{\frac{1}{2}}}$  \\[4pt]
		$\phantom{\dfrac{1}{\sqrt{2}}}=2^{-\frac{1}{2}}$
		\task $\sqrt[5]{4}=\left(2^2\right)^{\frac{1}{5}}$  \\[4pt]
		$\phantom{\sqrt[5]{4}}=2^{2\cdot\frac{1}{5}}$ \\[4pt]
		$\phantom{\sqrt[5]{4}}=2^{\frac{2}{5}}$
	\end{tasks}
\end{example}
\begin{exercise}
	Write as a power of two.
	\begin{tasks}(5)
		\task $\sqrt[5]{2}$
		\task $\dfrac{1}{\sqrt[5]{2}}$
		\task $2\sqrt{2}$
		\task $4\sqrt{2}$
		\task $\dfrac{1}{\sqrt[3]{2}}$
		\task $2\cdot\sqrt[3]{2}$
		\task $\frac{4}{\sqrt{2}}$
		\task $\left(\sqrt{2}\right)^3$
		\task $\dfrac{1}{\sqrt[3]{16}}$
		\task $\dfrac{1}{\sqrt{8}}$
	\end{tasks}
\end{exercise}
\begin{exercise}
	Write as a power of three.
	\begin{tasks}(5)
		\task $\sqrt[3]{3}$
		\task $\dfrac{1}{\sqrt[3]{3}}$
		\task $\sqrt[4]{2}$
		\task $3\sqrt{3}$
		\task $\dfrac{1}{9\sqrt{3}}$
	\end{tasks}
\end{exercise}
We can merge these rules into the previous ones simply by allowing rational exponents.
\begin{tcolorbox}
	For bases $a,b\in\mathbb R$, exponents $x,y\in\mathbb Q$ and $n\in\mathbb N$, we have the following rules:
	\begin{multicols}{3}
		\centering
		\textbf{same base}\\
		\begin{align*}
			a^x\cdot a^y&=a^{x+y} \\[10pt]
			\frac{a^x}{a^y}&=a^{x-y} \\[10pt]
			\left(a^x\right)^y&=a^{x\cdot y}
		\end{align*}
		\vfill
		\columnbreak
		
		\textbf{same exponent}\\
		\begin{align*}
			\left(a\cdot b\right)^x&=a^x\cdot b^x \\[10pt]
			\left(\frac{a}{b}\right)^x&=\frac{a^x}{b^x}
		\end{align*}
		\vfill
		\columnbreak
		
		\textbf{definitions}\\
		\begin{align*}
			a^0&=1 \\[10pt]
			a^{-x}&=\frac{1}{a^x} \\[10pt]
			\textcolor{blue}{a^{\frac{1}{n}}}&\textcolor{blue}{=\sqrt[n]{a}}
		\end{align*}
		\vfill
	\end{multicols}
	Note that $x$ and $y$ can now be fractions of integers, for example $x=\frac{m}{n}$.
\end{tcolorbox}