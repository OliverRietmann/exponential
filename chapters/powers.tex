\section*{Integer Exponents}
%In this section, we are going to develop rules for computations involving powers.
%These will enable us to handle exponential equations more efficiently.
\begin{tcolorbox}
	We consider the expression $a^n$. Then
	\begin{itemize}
		\item[] $\ldots$\ $a$ is called \textbf{base} and \ldots
		\item[] $\ldots$\ $n\in\mathbb Z$ is called \textbf{exponent} or \textbf{power}.
	\end{itemize}
\end{tcolorbox}
The expression $5^3$ is just an abbreviation of ``multiply 5 three times by itself '', i.e.
\begin{equation*}
	5^3=5\cdot 5\cdot 5=125.
\end{equation*}
More generally, for exponents $m,n\in\mathbb N$, we have
\begin{equation*}
	\textcolor{blue}{a}^m=\underbrace{\textcolor{blue}{a}\cdot\textcolor{blue}{a}\cdot\ldots\cdot\textcolor{blue}{a}}_{m\text{--times}}
	\qquad\text{and}\qquad
	\textcolor{purple}{a}^n=\underbrace{\textcolor{purple}{a}\cdot\textcolor{purple}{a}\cdot\ldots\cdot\textcolor{purple}{a}}_{n\text{--times}}.
\end{equation*}
Consequently,
\begin{equation*}
	\textcolor{blue}{a}^m\cdot\textcolor{purple}{a}^n
	=\underbrace{\textcolor{blue}{a}\cdot\textcolor{blue}{a}\cdot\ldots\cdot\textcolor{blue}{a}
	\cdot\textcolor{purple}{a}\cdot\textcolor{purple}{a}\cdot\ldots\cdot\textcolor{purple}{a}}_{(m+n)\text{--times}}
	=a^{m+n}.
\end{equation*}
We have thus found the following rule
\begin{equation*}
	a^m\cdot a^n=a^{m+n},
\end{equation*}
where $m,n\in\mathbb N$.
In fact, this rule holds even for more general exponents.
\begin{tcolorbox}
	For exponents $m,n\in\mathbb Z$, we have the following rules:
	\begin{multicols}{3}
		\centering
		\textbf{same base}\\
		\begin{align*}
			a^m\cdot a^n&=a^{m+n} \\[10pt]
			\frac{a^m}{a^n}&=a^{m-n} \\[10pt]
			\left(a^m\right)^n&=a^{m\cdot n}
		\end{align*}
		\vfill
		\columnbreak

		\textbf{same exponent}\\
		\begin{align*}
			\left(a\cdot b\right)^n&=a^n\cdot b^n \\[10pt]
			\left(\frac{a}{b}\right)^n&=\frac{a^n}{b^n}
		\end{align*}
		\vfill
		\columnbreak

		\textbf{only for $a\neq 0$}\\
		\begin{align*}
			a^0&=1 \\[10pt]
			a^{-n}&=\frac{1}{a^n}
		\end{align*}
		\vfill
	\end{multicols}
	Whenever negative powers are involved, the base must not be zero.
\end{tcolorbox}
The notation with negative exponents was proposed by the famous Swiss mathematician \textit{Leonhard Euler} in his book ``Vollständige Anleitung zur Algebra'' from 1770, where he justifies the notation using Table~\ref{tab:euler_powers}.
\begin{table}[ht]
	\centering
	\renewcommand{\arraystretch}{1.5}
	\begin{tabular}{|c|c|c|c|c|c|c|c|c|c|} \hline
		$\frac{1}{aaaaaa}$ & $\frac{1}{aaaaa}$ & $\frac{1}{aaaa}$ & $\frac{1}{aaa}$ & $\frac{1}{aa}$ & $\frac{1}{a}$ & 1 & $a$ & $aa$ & $aaa$ \\ \hline
		$\frac{1}{a^6}$ & $\frac{1}{a^5}$ & $\frac{1}{a^4}$ & $\frac{1}{a^3}$ & $\frac{1}{a^2}$ & $\frac{1}{a}$ & & & & \\ \hline
		$a^{-6}$ & $a^{-5}$ & $a^{-4}$ & $a^{-3}$ & $a^{-2}$ & $a^{-1}$ & $a^0$ & $a^1$ & $a^2$ & $a^3$ \\ \hline
	\end{tabular}
	\caption{Leonhard Euler, ``Vollständige Anleitung zur Algebra'', Page~66}
	\label{tab:euler_powers}
\end{table}
\begin{exercise}
	Write the following terms in the simplest form with prime base.
	\begin{tasks}(4)
		\task $9^3$
		\task $4\cdot 2^n$
		\task $\dfrac{3^m}{9^n}$
		\task $25^{x-1}$
	\end{tasks}
\end{exercise}
\begin{solution*}
	The the rules listed above, use the first column formulas (same base).
	\begin{tasks}(4)
		\task $3^8$
		\task $2^{n+2}$
		\task $3^{m-2n}$
		\task $5^{2x-2}$
	\end{tasks}
\end{solution*}
\begin{exercise}
	Write the following terms in the simplest form with prime base.
	\begin{tasks}(4)
		\task $8$ \task $25$ \task $27$ \task $4^3$
		\task $9^2$ \task $3^n\cdot 9$ \task $\dfrac{5^m}{5}$ \task $3^n\cdot 9^n$
		\task $\dfrac{16}{2^x}$ \task $\dfrac{3^{x+1}}{3^{x-1}}$ \task $\left(5^4\right)^{x-1}$ \task $2^x\cdot 2^{2-x}$
		\task $\dfrac{2^y}{4^x}$ \task $\dfrac{4^y}{8^x}$ \task $\dfrac{3^{x+1}}{3^{1-x}}$ \task $\dfrac{2^t\cdot 4^t}{8^{t-1}}$
	\end{tasks}
\end{exercise}