\section*{Integer Exponents}
%In this section, we are going to develop rules for computations involving powers.
%These will enable us to handle exponential equations more efficiently.
\begin{tcolorbox}
	We consider the expression $a^n$. Then
	\begin{itemize}
		\item[] $\ldots$\ $a$ is called \textbf{base} and \ldots
		\item[] $\ldots$\ $n\in\mathbb Z$ is called \textbf{exponent} or \textbf{power}.
	\end{itemize}
\end{tcolorbox}
The expression $5^3$ is just an abbreviation of ``multiply 5 three times by itself '', i.e.
\begin{equation*}
	5^3=5\cdot 5\cdot 5=125.
\end{equation*}
More generally, for exponents $m,n\in\mathbb N$, we have
\begin{equation*}
	\textcolor{blue}{a}^m=\underbrace{\textcolor{blue}{a}\cdot\textcolor{blue}{a}\cdot\ldots\cdot\textcolor{blue}{a}}_{m\text{--times}}
	\qquad\text{and}\qquad
	\textcolor{purple}{a}^n=\underbrace{\textcolor{purple}{a}\cdot\textcolor{purple}{a}\cdot\ldots\cdot\textcolor{purple}{a}}_{n\text{--times}}.
\end{equation*}
Consequently,
\begin{equation*}
	\textcolor{blue}{a}^m\cdot\textcolor{purple}{a}^n
	=\underbrace{\textcolor{blue}{a}\cdot\textcolor{blue}{a}\cdot\ldots\cdot\textcolor{blue}{a}
	\cdot\textcolor{purple}{a}\cdot\textcolor{purple}{a}\cdot\ldots\cdot\textcolor{purple}{a}}_{(m+n)\text{--times}}
	=a^{m+n}.
\end{equation*}
We have thus found the following rule
\begin{equation*}
	a^m\cdot a^n=a^{m+n},
\end{equation*}
where $m,n\in\mathbb N$.
In fact, this rule holds even for more general exponents.
\annotation{5}
\begin{tcolorbox}
	For exponents $m,n\in\mathbb Z$, we have the following rules:
	\begin{multicols}{3}
		\centering
		\textbf{same base}\\
		\begin{align*}
			a^m\cdot a^n&=a^{m+n} \\[10pt]
			\frac{a^m}{a^n}&=a^{m-n} \\[10pt]
			\left(a^m\right)^n&=a^{m\cdot n}
		\end{align*}
		\vfill
		\columnbreak

		\textbf{same exponent}\\
		\begin{align*}
			\left(a\cdot b\right)^n&=a^n\cdot b^n \\[10pt]
			\left(\frac{a}{b}\right)^n&=\frac{a^n}{b^n}
		\end{align*}
		\vfill
		\columnbreak

		\textbf{definitions}\\
		\begin{align*}
			a^0&=1 \\[10pt]
			a^{-n}&=\frac{1}{a^n}
		\end{align*}
		\vfill
	\end{multicols}
	Whenever negative powers are involved, the base must not be zero.
\end{tcolorbox}
The notation with negative exponents was proposed by the famous Swiss mathematician \textit{Leonhard Euler} in his book ``Vollständige Anleitung zur Algebra'' from 1770, where he justifies the notation using Table~\ref{tab:euler_powers}.
\begin{table}[ht]
	\centering
	\renewcommand{\arraystretch}{1.5}
	\begin{tabular}{|c|c|c|c|c|c|c|c|c|c|} \hline
		$\frac{1}{aaaaaa}$ & $\frac{1}{aaaaa}$ & $\frac{1}{aaaa}$ & $\frac{1}{aaa}$ & $\frac{1}{aa}$ & $\frac{1}{a}$ & 1 & $a$ & $aa$ & $aaa$ \\ \hline
		$\frac{1}{a^6}$ & $\frac{1}{a^5}$ & $\frac{1}{a^4}$ & $\frac{1}{a^3}$ & $\frac{1}{a^2}$ & $\frac{1}{a}$ & & & & \\ \hline
		$a^{-6}$ & $a^{-5}$ & $a^{-4}$ & $a^{-3}$ & $a^{-2}$ & $a^{-1}$ & $a^0$ & $a^1$ & $a^2$ & $a^3$ \\ \hline
	\end{tabular}
	\caption{Leonhard Euler, ``Vollständige Anleitung zur Algebra'', Page~66}
	\label{tab:euler_powers}
\end{table}
\annotation{10}
\begin{example}
	We write the term on the left in the simplest form with prime base.
	\begin{tasks}(4)
		\task $9^4=\left(3^2\right)^4$ \\[4pt]
		$\phantom{9^4}=3^{2\cdot 4}$ \\[4pt]
		$\phantom{9^4}=3^8$
		\task $4\cdot 2^n=2^2\cdot 2^n$ \\[4pt]
		$\phantom{4\cdot 2^n}=2^{2+n}$
		\task $\dfrac{3^m}{9^n}=\dfrac{3^m}{\left(3^2\right)^n}$ \\[4pt]
		$\phantom{\dfrac{3^m}{9^n}}=\dfrac{3^m}{3^{2n}}$ \\[4pt]
		$\smash{\phantom{\dfrac{3^m}{9^n}}}=3^{m-2n}$
		\task $25^{x-1}=\left(5^2\right)^{x-1}$ \\[4pt]
		$\phantom{25^{x-1}}=5^{2\left(x-1\right)}$ \\[4pt]
		$\phantom{25^{x-1}}=5^{2x-2}$
	\end{tasks}
\end{example}
\annotation{15}
\begin{exercise}
	Write the following terms in the simplest form with prime base.
	\begin{tasks}(4)
		\task $8$ \task $25$ \task $27$ \task $4^3$
		\task $9^2$ \task $3^n\cdot 9$ \task $\dfrac{5^m}{5}$ \task $3^n\cdot 9^n$
		\task $\dfrac{16}{2^x}$ \task $\dfrac{3^{x+1}}{3^{x-1}}$ \task $\left(5^4\right)^{x-1}$ \task $2^x\cdot 2^{2-x}$
		\task $\dfrac{2^y}{4^x}$ \task $\dfrac{4^y}{8^x}$ \task $\dfrac{3^{x+1}}{3^{1-x}}$ \task $\dfrac{2^t\cdot 4^t}{8^{t-1}}$
	\end{tasks}
\end{exercise}
\annotation{35}
\begin{example}
	We write the term on the left in the simplest form without brackets.
	\begin{tasks}(2)
		\task $\left(2x\right)^3=2^3\cdot x^3$ \\ $\phantom{\left(2x\right)^3}=8x^3$
		\task $\left(\dfrac{3c}{b}\right)^4=\dfrac{3^4\cdot c^4}{b^4}$ \\
		$\phantom{\left(\dfrac{3c}{b}\right)^4}=\dfrac{81c^4}{b^4}$
	\end{tasks}
\end{example}
\annotation{38}
\begin{exercise}
	Write the following terms in the simplest form without brackets.
	\begin{tasks}(4)
		\task $\left(2b^4\right)^3$ \task $\left(\dfrac{3}{x^2y}\right)^2$
		\task $\left(5a^4b\right)^2$ \task $\left(\dfrac{m^3}{2n^2}\right)^4$
		\task $\left(\dfrac{3a^3}{b^5}\right)^3$ \task $\left(2m^3n^2\right)^5$
		\task $\left(\dfrac{4a^4}{b^2}\right)^2$ \task $\left(5x^2y^3\right)^3$
		\task $\left(-2a\right)^2$ \task $\left(-6b^2\right)^2$
		\task $\left(-2a\right)^3$ \task $\left(-3m^2n^2\right)^3$
		\task $\left(-2ab^4\right)^4$ \task $\left(\dfrac{-2a^2}{b^2}\right)^3$
		\task $\left(\dfrac{-4a^3}{b}\right)^2$ \task $\left(\dfrac{-3p^2}{q^3}\right)^2$
	\end{tasks}
\end{exercise}
\annotation{55}
\begin{exercise}
	Derive the following two laws of exponents for $n,m\in\mathbb N$.
	\begin{tasks}(2)
		\task $\left(a^m\right)^n=a^{m\cdot n}$
		\task $\left(a\cdot b\right)^n=a^n\cdot b^n$
	\end{tasks}
\end{exercise}
\annotation{65}
\pagebreak[4]
\begin{tcolorbox}
	\textbf{Exponential equations:} The unknown $x$ occurs in the exponent.
	\begin{itemize}
		\item exponential equation: $2^x=16$
		\item \textbf{not} an exponential equation: $x^2=16$
	\end{itemize}
	Solving exponential equations:
	\begin{enumerate}[Step 1:]
		\item Express both sides as in terms of the same base.
		\item Compare the exponents.
	\end{enumerate}
\end{tcolorbox}
\begin{example}
	We follow these two steps.\\
	\begin{minipage}{0.45\textwidth}
		\begin{align*}
			\textrm{a)}\quad 2^x&=16 \\
			2^x&=2^4 \\
			x&=4 \\
			\phantom{x+2}& \\
			\phantom{x+2}&
		\end{align*}
	\end{minipage}\hfill
	\begin{minipage}{0.45\textwidth}
		\begin{align*}
			\textrm{b)}\quad 3^{x+2}&=\frac{1}{27} \\
			3^{x+2}&=3^{-3} \\
			x+2&=-3 \\
			x&=-5
		\end{align*}
	\end{minipage}
\end{example}
\annotation{70}
\begin{exercise}
	Solve for $x$.
	\begin{tasks}(4)
		\task $2^x=2$ \task $2^x=4$ \task $3^x=27$ \task $2^x=1$
		\task $2^x=\frac{1}{2}$ \task $3^x=\frac{1}{3}$ \task $2^x=\frac{1}{8}$ \task $2^{x+1}=8$
		\task $2^{x-2}=\frac{1}{4}$ \task $3^{x+1}=\frac{1}{27}$ \task $2^{x+1}=64$
		\task $2^{1-2x}=\frac{1}{2}$
	\end{tasks}
\end{exercise}
\annotation{85}
\vspace*{2cm}
\hrule
\begin{center} \textbf{Further Examples and Exercises} \end{center}
\begin{example}
	Write the terms on the left in the simplest form.
	\begin{tasks}(3)
		\task $3x^2\cdot 5x^5=3\cdot 5\cdot x^2\cdot x^5$ \\
		$\phantom{3x^2\cdot 5x^5}=15x^{2+5}$ \\
		$\phantom{3x^2\cdot 5x^5}=15x^7$
		\task $\dfrac{20a^9}{4a^6}=\dfrac{20}{4}\cdot a^{9-6}$ \\
		$\smash{\phantom{\dfrac{20a^9}{4a^6}}}=5a^3$
		\task $\dfrac{b^3\cdot b^7}{\left(b^2\right)^4}=\dfrac{b^{10}}{b^8}$ \\
		$\smash{\phantom{\dfrac{b^3\cdot b^7}{\left(b^2\right)^4}}}=b^{10-8}$ \\
		$\smash{\phantom{\dfrac{b^3\cdot b^7}{\left(b^2\right)^4}}}=b^2$ 
	\end{tasks}
\end{example}
\annotation{5}
\begin{exercise}
	Write the following terms in the simplest form without brackets.
	\begin{tasks}(3)
		\task $\dfrac{a^3}{a}$
		\task $4b^2\cdot 2b^3$
		\task $\dfrac{m^5n^4}{m^2n^3}$
		\task $\dfrac{14a^7}{2a^2}$
		\task $\dfrac{12a^2b^3}{3ab}$
		\task $\dfrac{18m^7a^3}{4m^4a^3}$
	\end{tasks}
\end{exercise}
\annotation{15}
\begin{exercise}
	Solve for $x$.
	\begin{tasks}(4)
		\task $4^x=32$ \task $8^x=\frac{1}{4}$ \task $9^x=\frac{1}{3}$ \task $49^x=\frac{1}{7}$
		\task $4^x=\frac{1}{8}$ \task $25^x=\frac{1}{5}$ \task $8^{x+2}=32$
		\task $8^{1-x}=\frac{1}{4}$
		\task $4^{2x-1}=\frac{1}{2}$ \task $9^{x-3}=3$ \task $\left(\frac{1}{2}\right)^{x+1}=2$
		\task $\left(\frac{1}{3}\right)^{x+2}=9$
		\task $4^x=8^{-x}$ \task $\left(\frac{1}{4}\right)^{1-x}=8$
		\task $\left(\frac{1}{7}\right)^x=49$ \task $\left(\frac{1}{2}\right)^{x+1}=32$ 
	\end{tasks}
\end{exercise}
\annotation{30}